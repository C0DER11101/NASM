\documentclass{article}
\usepackage{hyperref}
\usepackage{graphicx}
\usepackage{amsmath}
\usepackage{fancyvrb}

\hypersetup{
	colorlinks=true,
	linkcolor=red,
	urlcolor=blue
}

\makeatletter

\renewcommand\paragraph{\@startsection{paragraph}{4}{\z@}{-3.25ex \@plus -1ex \@minus -.2ex}{1.5ex \@plus .2ex}{\normalfont\normalsize\bfseries}}

\makeatother

\begin{document}
\section{Dividing integers with \textit{DIV} and \textit{IDIV}}\label{sec:sec1}
\paragraph{}
\begin{itemize}
\item \textbf{DIV} $\rightarrow$ Divides unsigned numbers.
\item \textbf{IDIV} $\rightarrow$ Divides signed numbers.
\end{itemize}

\subsection{The DIV instruction}\label{sec:subsec1}
\paragraph{}
Consider the following assembly program:
\begin{Verbatim}[numbers=left, frame=single]
section .data

section .text
	global _start
	
	_start:
		MOV eax, 11
		MOV ecx, 2
		
		DIV ecx
		
		MOV eax, 1
		INT 80h
\end{Verbatim}
The \textbf{DIV} and \textbf{IDIV} instructions work just like the \textbf{MUL} and \textbf{IMUL}.
The \textbf{eax} register is automatically used by the \textbf{DIV} and \textbf{IDIV} instructions.

In the program, \textbf{eax} is the dividend and \textbf{ecx} is the divisor. The quotient gets stored in \textbf{eax} and the remainder gets stored in the register \textbf{edx} as shown in the image below:

\begin{figure}[h]
\centering
\includegraphics[width=0.5\textwidth]{Output53.png}
\caption{Values in \textbf{eax} and \textbf{edx}}
\label{fig:fig1}
\end{figure}
\newpage
\subsection{The IDIV instruction}\label{sec:subsec2}
\paragraph{}
Program:
\begin{Verbatim}[numbers=left, frame=single]
section .data

section .text
	MOV eax, -6
	MOV ecx, 2
	
	IDIV ecx
	
	MOV eax, 1
	INT 80h
\end{Verbatim}
Here, we are dividing a negative number($-6$) by $2$.

These values are stored in the registers \textbf{eax} and \textbf{ebx}:
\begin{figure}[h]
\centering
\includegraphics[width=0.6\textwidth]{Output54.png}
\caption{Values of \textbf{eax} before and after dividing.}
\label{fig:fig2}
\end{figure}

So, we can see that \textbf{edx} stores the remainder(as we know) which is $0$. \textbf{eax} stores the the quotient. We can see that \textbf{eax} stores a rather larger value, it was expected to store $-3$.

Now, if we check the value in the sub-register \textbf{ax} then we get $-3$:
\begin{figure}[h]
\centering
\includegraphics[width=0.5\textwidth]{Output55.png}
\caption{$-3$ in \textbf{ax}}
\label{fig:fig3}
\end{figure}

\newpage
Now, we also check in the registers \textbf{ah} and \textbf{al}. We get the following coutput:
\begin{figure}[h]
\centering
\includegraphics[width=0.5\textwidth]{Output56.png}
\caption{Values in \textbf{ah} and \textbf{al}}
\label{fig:fig4}
\end{figure}

\textbf{al} stores $-3$ and \textbf{ah} stores $-1$.

On converting the value of \textbf{eax} to binary we get this:

$$
01111111\ 11111111\ 11111111\ 11111101
$$
\textbf{ax} stores the last $16$-bits, out of those $16$-bits, the higher $8$-bits are in \textbf{ah} and the lower $8$-bits are stored in \textbf{al}. We also know that $11111101$ is $-3$.

\subsubsection{Comparision}\label{sec:subsubsec1}
\paragraph{}
If we take $-6$ and $-3$, this how their binary representation is in $32$-bit:

$\bullet -6$

$$
11111111\ 11111111\ 11111111\ 11111010
$$

$\bullet -3$

$$
11111111\ 11111111\ 11111111\ 11111101
$$

Notice that in the binary representation of $-3$ and $-6$ the last $0$ in $-6$ is missing from $-3$. It almost looks like $-6$ was shifted towards the right by $1$ bit(but that's not the case here).

Now, when we were debugging our program, \textbf{eax} had the following values in it. Binary representation has also been provided(in $32$-bit binary):

$1^{st}$ value in \textbf{eax}: $-6$
$$
11111111\ 11111111\ 11111111\ 11111010
$$

$2^{nd}$ value in \textbf{eax}: $2147483645$
$$
01111111\ 11111111\ 11111111\ 11111101
$$
Now, here the second value in \textbf{eax} seems to be a result of shifting $-6$(the previous value of \textbf{eax}) to the right by $1$-bit.
\end{document}