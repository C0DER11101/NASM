\documentclass{article}
\usepackage{xr}
\usepackage{enumitem}
\usepackage{fancyvrb}
\usepackage{hyperref}
\usepackage{graphicx}
\usepackage{varwidth}

\externaldocument{note1}

\hypersetup{
	colorlinks=true,
	linkcolor=red,
	urlcolor=blue
}

\makeatletter

\renewcommand\paragraph{\@startsection{paragraph}{4}{\z@}{-3.25ex \@plus -1ex \@minus -.2ex}{1.5ex \@plus .2ex}{\normalfont\normalsize\bfseries}}

\makeatother


\begin{document}
\section{Data section}\label{sec:dSection}
\paragraph{}
We define variables inside the \textbf{data} section of the assembly program.

The format is like so:
\begin{Verbatim}[frame=single]
section .data
	num DD 5
section .text
	global _start
	....
	....
	....
\end{Verbatim}
So while declaring variables in the \textbf{data} section, provide these three things:
\begin{enumerate}
\item Name of the variable(in this case it's \textbf{num}).
\item Type of the variable(in this case it's \textbf{DD}).
\item Value for the variable(in this case it's $5$).
\end{enumerate}

Here are the various types that we can give to variables:
\begin{enumerate}
\item \textbf{DB} $\rightarrow$ Define byte($1$ byte).
\item \textbf{DW} $\rightarrow$ Define word($2$ bytes).
\item \textbf{DD} $\rightarrow$ Double word($4$ bytes).
\item \textbf{DQ} $\rightarrow$ Double-precision floating-point constants($8$ bytes).
\item \textbf{DT} $\rightarrow$ Extended-precision floating-point constants($10$ bytes).
\end{enumerate}

This is the complete program:

\begin{Verbatim}[numbers=left, frame=single]
section .data
	num DD 5
section .text
	global _start
	
	_start:
		MOV eax, 1
		MOV ebx, num
		INT 80h
\end{Verbatim}

Here, we tried to move the value of \textbf{num} into the register \textbf{ebx}.
\newpage
Notice what happens when we run this program:

\textbf{Compilation}
\begin{Verbatim}[frame=single]
$ nasm -f elf -o asm2.o asm2.asm
\end{Verbatim}
\textbf{Linking}
\begin{Verbatim}[frame=single]
$ ld -m elf_i386 -o asm2 asm2.o
\end{Verbatim}

Now, we will run the executable:

\begin{Verbatim}[frame=single]
$ ./asm2
\end{Verbatim}

The program runs without any errors, but what value does \textbf{\$?} store????

\begin{figure}[h]
\centering
\includegraphics[width=0.7\textwidth]{Output9.png}
\caption{Unexpected value of \textbf{\$?}}
\end{figure}

We can see that \textbf{\$?} stores $0$ when it should have stored $5$ in it because we stored the value \textbf{num} into \textbf{ebx}. \textbf{DIDN'T WE??}

\section{Debugging time}\label{sec:debugging}
\paragraph{}
Let's see what GDB has to tell us.
Everything is the same as \autoref{sec:gdbexec}.
\newpage
In the figure below:
\begin{figure}[h]
\centering
\includegraphics[width=0.5\textwidth]{Output10.png}
\caption{Surprise!}
\label{fig:surprise}
\end{figure}

we can see a value getting stored into the register \textbf{ebx} but we are sure that it's not $5$ because when we use the \textbf{info registers} command we can see the decimal form of that hexadecimal number and we can see that it's not $5$. 

\textit{What is it??} That is actually the address that is getting stored into register \textbf{ebx}.

\textit{Whose address??} The address of the data for the variable \textbf{num}. This address is stored by the variable \textbf{num}. So, \textbf{num} is basically storing the location on the stack where the number $5$ is located. Oh no! Where is $5$!!! No worries, we can view the value $5$ via the following command:

\begin{Verbatim}[frame=single]
(gdb) x/x $ebx
\end{Verbatim}
This will give us the following result:
\begin{figure}[h]
\centering
\includegraphics[width=0.5\textwidth]{Output11.png}
\caption{Value at address stored in \textbf{ebx}}
\end{figure}

\newpage
Wow! It's the same hexadecimal address \textbf{0x804a000} but there is also another value in hexadecimal along with it and that is $5$. So this tells us that the location(\textbf{0x804a000} is where $5$ is located.

\subsection{Then how do we get the value $5$ out of that address??}\label{sec:howto}
\paragraph{}

Following changes are made:
\begin{Verbatim}[frame=single]
MOV ebx, [num]
\end{Verbatim}
The $\lbrack$ $\rbrack$ are kind of like the dereference operator(*) in C/C++. Basically what it does is that it goes to the address stored in \textbf{num} gets the value stored in that address and moves it to the register \textbf{ebx}.

\begin{figure}[h]
\centering
\includegraphics[width=0.7\textwidth]{Output12.png}
\caption{Value of \textbf{\$?} as expected}
\end{figure}

Let's see with GDB.

\begin{figure}[h]
\centering
\includegraphics[width=0.5\textwidth]{Output13.png}
\caption{Notice the difference}
\end{figure}

Notice the difference between this figure and figure \ref{fig:surprise}. There is no \$ before \textbf{0x804a000} in this figure but there is one on the other figure.

\newpage
Also, the command \textbf{info registers} command also gives the following output:
\begin{figure}[h]
\centering
\includegraphics[width=0.5\textwidth]{Output14.png}
\caption{\textbf{ebx} stores $5$}
\end{figure}

\end{document}