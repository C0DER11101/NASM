\documentclass{article}
\usepackage{hyperref}
\usepackage{graphicx}
\usepackage{amsmath}
\usepackage{fancyvrb}

\hypersetup{
	colorlinks=true,
	linkcolor=red,
	urlcolor=blue
}

\makeatletter

\renewcommand\paragraph{\@startsection{paragraph}{4}{\z@}{-3.25ex \@plus -1ex \@minus -.2ex}{1.5ex \@plus .2ex}{\normalfont\normalsize\bfseries}}

\makeatother

\begin{document}
\section{Multiplying numbers with \textit{MUL} and \textit{IMUL}}\label{sec:sec1}
\paragraph{}

\begin{itemize}
\item \textbf{MUL} $\rightarrow$ Multiplies unsigned numbers.

\item \textbf{IMUL} $\rightarrow$ Multiplies signed numbers.
\end{itemize}

\subsection{Using the \textbf{MUL} instruction}\label{sec:subsec1}
\paragraph{}
Assembly program to multiply two numbers using \textbf{MUL}:
\begin{Verbatim}[numbers=left, frame=single]
section .data

section .text
	global _start
	_start:
		MOV al, 2
		MOV bl, 3
		MUL bl
		
		MOV eax, 1
		INT 80h
\end{Verbatim}
The \textbf{MUL} instruction requires only one register. The reason is that the register \textbf{eax} is automatically used by multiplication and division instructions meaning that we don't need to specify it explicitly.

Here is an image showing the value in \textbf{al} before and after multiplication:
\begin{figure}[h]
\centering
\includegraphics[width=0.6\textwidth]{Output49.png}
\caption{Values of \textbf{al}}
\label{fig:fig1}
\end{figure}

\newpage
\underline{What happens if the result of the multiplication is larger than the register?}

\textbf{Program:}
\begin{Verbatim}[numbers=left, frame=single]
section .data

section .text
	global _start
	_start:
		MOV al, 0xFF
		MOV bl, 2
		MUL bl
		
		MOV eax, 1
		INT 80h
\end{Verbatim}
\textbf{0xFF} in $8$-bit representation is $-1$ and in $16$-bit representation is $255$. Now, we store this value in the register \textbf{al}.

When the multiplication operations is performed then we get the following result:
\begin{figure}[h]
\centering
\includegraphics[width=0.6\textwidth]{Output50.png}
\caption{Negative numbers}
\label{fig:fig2}
\end{figure}

So after performing multiplication we see that $-2$ is stored in \textbf{al}. We know that $-2$ can be represented in $8$-bit binary as $11111110$.
\newpage
Now, if we check what is stored in the higher-bit register \textbf{ah}, then we will get the following:
\begin{figure}[h]
\centering
\includegraphics[width=0.6\textwidth]{Output51.png}
\caption{Value in \textbf{ah} and $16$-bit representation of $-2$}
\label{fig:fig3}
\end{figure}

Now, after that we check what is the value stored in $16$-bits that is register \textbf{ax}. We get $510$.

Now the binary representation of $510$ in $16$-bits is:
$$
00000001\ 11111110
$$
The lower $8$-bits are in \textbf{al} and higher $8$-bits are in the register \textbf{ah}.
\newpage
\subsection{Using the IMUL instruction}\label{sec:subsec2}
\paragraph{}
Program:
\begin{Verbatim}[numbers=left, frame=single]
section .data

section .text
	global _start
	
	_start:
		MOV al, 0xFF
		MOV bl, 2
		IMUL bl
		
		MOV eax, 1
		INT 80h
\end{Verbatim}
When we multiply two $n$-bit numbers, the product is $2*n$-bits long. Keeping that in mind, when $2$ is multiplied to $-1$ then we get the product as $-2$. This product is stored in the register \textbf{al} and also in \textbf{ah} because the product is $16$(i.e $2*8$)-bits long.

This is shown in the image below:
\begin{figure}[h]
\centering
\includegraphics[width=0.6\textwidth]{Output52.png}
\caption{Registers: \textbf{al}, \textbf{ah} and \textbf{ax}}
\label{fig:fig4}
\end{figure}

Basically this is what is stored in \textbf{ax}:
$$
11111111\ 11111110
$$
The lower $8$-bits are stored in \textbf{al} and the higher $8$-bits are stored in \textbf{ah}.

\end{document}