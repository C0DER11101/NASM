\documentclass{article}
\usepackage{hyperref}
\usepackage{graphicx}
\usepackage{amsmath}
\usepackage{fancyvrb}

\hypersetup{
	colorlinks=true,
	linkcolor=red,
	urlcolor=blue
}

\makeatletter

\begin{document}
\section{Logical operators}\label{sec:sec1}
\subsection{The AND, OR and NOT operators}\label{sec:subsec1}
\paragraph{}
Assembly program:
\begin{Verbatim}[numbers=left, frame=single]
section .data

section .text
	global _start
	_start:
		MOV eax, 0b1010
		MOV ebx, 0b1100
		
		AND eax, ebx   ; AND operation
		
		MOV eax, 0b1010
		MOV ebx, 0b1100
		
		OR eax, ebx    ; OR operation
		
		NOT eax        ; NOT operation
		
		MOV eax, 1
		INT 80h
\end{Verbatim}
The value in \textbf{eax} is $10$ and in \textbf{ebx} is $12$. The first operation is the \textbf{AND} operation. So, when we perform \textbf{AND} between $10$ and $12$ i.e

\begin{align*}
00000000\ 00000000\ 00000000\ 00001010 \\
AND\ 00000000\ 00000000\ 00000000\ 00001100 \\
\hline
00000000\ 00000000\ 00000000\ 00001000 \\
\end{align*}
which results in $8$. This result is stored in \textbf{eax} as shown in the image below:
\begin{figure}[h]
\centering
\includegraphics[width=0.43\textwidth]{Output57.png}
\caption{\textbf{AND} operation}
\label{fig:fig1}
\end{figure}
\newpage
The next operation is \textbf{OR} and we re-assign the values $10$ and $12$ to \textbf{eax} and \textbf{ebx} respectively.

So, we perform the \textbf{OR} operation between \textbf{eax} and \textbf{ebx} i.e

\begin{align*}
00000000\ 00000000\ 00000000\ 00001010 \\
OR\ 00000000\ 00000000\ 00000000\ 00001100 \\
\hline
00000000\ 00000000\ 00000000\ 00001110 \\
\end{align*}
which results in $14$ and again stored in \textbf{eax} as shown below:
\begin{figure}[h]
\centering
\includegraphics[width=0.5\textwidth]{Output58.png}
\caption{\textbf{OR} operation}
\label{fig:fig2}
\end{figure}

The final operation is the \textbf{NOT} operation. Here we basically invert all the set bits and the unset bits. The value in \textbf{eax} is $14$(from the previous \textbf{OR} operation).
\begin{align*}
NOT\ 00000000\ 00000000\ 00000000\ 00001110 \\
\hline
11111111\ 11111111\ 11111111\ 11110001 \\
\end{align*}
This value is $-15$ which gets stored in \textbf{eax}.
\begin{figure}[h]
\centering
\includegraphics[width=0.5\textwidth]{Output59.png}
\caption{\textbf{NOT} operation}
\label{fig:fig3}
\end{figure}

\newpage
\section{Masking}\label{sec:sec2}
\paragraph{}
Let's say, we have a value $10$ in \textbf{eax} and after performing a logical operation, we want only the last $4$-bits to be affected by the operation. Then we would use a mask. We need to \textbf{filter} those bits that we need.

\textbf{Program:}
\begin{Verbatim}[numbers=left, frame=single]
section .data

section .text
	global _start
	
	_start:
		MOV eax, 0b1010
		NOT eax
		
		AND eax, 0x0000000f
		
		MOV eax, 1
		INT 80h
\end{Verbatim}

If we perform a \textbf{NOT} operation on \textbf{eax} which stores $10$ then this is what we will get:
\begin{align*}
NOT\ 00000000\ 00000000\ 00000000\ 00001010 \\
\hline
11111111\ 11111111\ 11111111\ 11110101 \\
\end{align*}
is $-11$ which is stored in \textbf{eax} as shown below:
\begin{figure}[h]
\centering
\includegraphics[width=0.5\textwidth]{Output60.png}
\caption{$-11$ stored in \textbf{eax}}
\label{fig:fig4}
\end{figure}

We only wanted the higher $24$-bits to change. We will use a mask that will only change the higher $24$-bits and will keep the lower $4$-bits unchanged.

\textbf{0x0000000f} is a $32$-bit mask whose last $4$-bits are set. First we performed the \textbf{NOT} operation on \textbf{eax} and then we performed the \textbf{AND} operation between \textbf{eax} and the mask.
\newpage
So when we perform \textbf{AND} operation:
\begin{align*}
11111111\ 11111111\ 11111111\ 11110101 \\
AND\ 00000000\ 00000000\ 00000000\ 00001111 \\
\hline
00000000\ 00000000\ 00000000\ 00000101 \\
\end{align*}
We get $5$ as the result.

Image:
\begin{figure}[h]
\centering
\includegraphics[width=0.5\textwidth]{Output61.png}
\caption{Keeping the last $4$ bits unchanged}
\label{fig:fig5}
\end{figure}
\newpage
\section{The XOR operation}\label{sec:sec3}
\paragraph{}
\textbf{Program:}
\begin{Verbatim}[numbers=left, frame=single]
section .data

section .text
	global _start
	_start:
		MOV eax, 0b1010
		MOV ebx, 0b1100
		
		XOR eax, ebx
		
		MOV eax, 1
		INT 80h
\end{Verbatim}
We are performing \textbf{XOR} operation:
\begin{align*}
00000000\ 00000000\ 00000000\ 00001010 \\
XOR\ 00000000\ 00000000\ 00000000\ 00001100 \\
\hline
00000000\ 00000000\ 00000000\ 00000110 \\
\end{align*}
We get $6$ which is stored in \textbf{eax} as shown below:
\begin{figure}[h]
\centering
\includegraphics[width=0.5\textwidth]{Output62.png}
\caption{\textbf{XOR}}
\label{fig:fig6}
\end{figure}
\end{document}