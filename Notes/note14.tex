\documentclass{article}
\usepackage{hyperref}
\usepackage{graphicx}
\usepackage{amsmath}
\usepackage{fancyvrb}

\hypersetup{
	colorlinks=true,
	linkcolor=red,
	urlcolor=blue
}

\makeatletter

\renewcommand\paragraph{\@startsection{paragraph}{4}{\z@}{-3.25ex \@plus -1ex \@minus -.2ex}{1.5ex \@plus .2ex}{\normalfont\normalsize\bfseries}}

\makeatother

\begin{document}
	\section{Comparing floating point numbers}\label{sec:sec1}
	\paragraph{}
	Consider this program:
	
	\begin{Verbatim}[numbers=left, frame=single]
		section .data
			x DD 3.14
			y DD 2.1
			
		section .text
			global _start
			_start:
				MOVSS xmm0, [x]
				MOVSS xmm1, [y]
				
				UCOMISS xmm0, xmm1
				JA greater
				JMP end
				
			greater:
				MOV eax, 1
				INT 80h
				
			end:
				MOV eax, 1
				INT 80h
	\end{Verbatim}

Here, we are comparing two floating point numbers. For comparision between floating point numbers we donot use the \textbf{CMP} instruction, instead we use the \href{https://www.felixcloutier.com/x86/ucomiss}{\textbf{UCOMISS}} instruction. This works the same way as the \textbf{CMP} instruction works, it also sets some eflags based on the comparisions.

\vspace{10pt}
The jumps that we perform with floating point comparisions are also different. Here we donot perform \textbf{JGE}, or \textbf{JGT}, etc... We instead use instructions \textbf{JB}(\textbf{J}ump \textbf{B}elow), \textbf{JA}(\textbf{J}ump \textbf{A}bove).
\newpage
There are many jump instructions:
\begin{itemize}
	\item \textbf{JE $<$label$>$} $\rightarrow$ \textbf{if op1 $==$ op2}
	\item \textbf{JNE $<$label$>$} $\rightarrow$ \textbf{if op1 $!=$ op2}
	\item \textbf{JB $<$label$>$} $\rightarrow$ \textbf{if op1 $<$ op2}
	\item \textbf{JBE $<$label$>$} $\rightarrow$ \textbf{if op1 $\leq$ op2}
	\item \textbf{JA $<$label$>$} $\rightarrow$ \textbf{if op1 $>$ op2}
	\item \textbf{JAE $<$label$>$} $\rightarrow$ \textbf{if op1 $\geq$ op2}
\end{itemize}

\textbf{Source: } \href{https://eng.libretexts.org/Bookshelves/Computer_Science/Programming_Languages/x86-64_Assembly_Language_Programming_with_Ubuntu_(Jorgensen)/18%3A_Floating-Point_Instructions/18.06%3A_Section_6-#:~:text=The%20conditional%20control%20instructions%20include,above%20or%20equal%20(jae).}{Click here}.

But we can still use the \textbf{JMP} instruction which jumps to a label unconditionally.
\end{document}