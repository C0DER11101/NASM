\documentclass{article}
\usepackage{hyperref}
\usepackage{graphicx}
\usepackage{listings}
\usepackage{amsmath}
\usepackage{fancyvrb}

\hypersetup{
	colorlinks=true,
	linkcolor=red,
	urlcolor=blue
}

\makeatletter

\renewcommand\paragraph{\@startsection{paragraph}{4}{\z@}{-3.25ex \@plus -1ex \@minus -.2ex}{1.5ex \@plus .2ex}{\normalfont\normalsize\bfseries}}

\makeatother

\begin{document}
	\section{Calling C functions}\label{sec:sec1}
	\paragraph{}
	Assembly program:
	\begin{lstlisting}[frame=single, breaklines=true]
		extern test
		extern exit
		
		section .text
			global main
			
			main:
				PUSH 1
				PUSH 2
				
				CALL test
				
				PUSH eax
				CALL exit
	\end{lstlisting}
Also, the \textbf{test()} function is defined in \textbf{test.c} as:
\begin{lstlisting}[frame=single, breaklines=true]
	int test(int a, int b){
		printf("This is test()\n");
		return (a+b);
	}
\end{lstlisting}
We \textbf{PUSH} $1$ and $2$ into the stack($2$ is the first argument of \textbf{test()} and $1$ is the second argument). After this we \textbf{CALL} the \textbf{test()} function with the arguments(from the stack). Now, \textbf{test()} returns the sum of the arguments passed to it. \underline{\textbf{The return value is stored in the register:}} \textbf{\textit{eax}}.
So we can \textbf{PUSH} the value of \textbf{eax} onto the stack and \textbf{CALL} \textbf{exit()}.

Now comes the compilation part, so first we use the nasm command to assemble our program:
\begin{lstlisting}[frame=single, breaklines=true]
	$ nasm -f elf -o asm16.o asm16.asm
\end{lstlisting}

Next, compile using \textbf{gcc}:
\begin{lstlisting}[frame=single, breaklines=true]
	$ gcc -no-pie -m32 asm16.o test.c -o asm16
\end{lstlisting}
Here we have to provide the \textbf{C} file where we have defined our \textbf{test()} function.
\newpage
We pushed \textbf{eax} into the stack and called \textbf{exit}. Now \textbf{exit} will execute as \textbf{exit(the value in eax)} and that value in \textbf{eax} will get stored in \textbf{\$?} variable which we can see from the image below:
\begin{figure}[h]
	\centering
	\includegraphics[width=1.2\textwidth]{Output96.png}
	\caption{Output}
	\label{fig:fig1}
\end{figure}
\end{document}