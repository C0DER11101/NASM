\documentclass{article}
\usepackage{fancyvrb}
\usepackage{amsmath}
\usepackage{hyperref}
\usepackage{graphicx}

\hypersetup{
	colorlinks=true,
	linkcolor=red,
	urlcolor=blue
}

\makeatletter

\renewcommand\paragraph{\@startsection{paragraph}{4}{\z@}{-3.25ex \@plus -1ex \@minus -.2ex}{1.5ex \@plus .2ex}{\normalfont\normalsize\bfseries}}

\makeatother

\begin{document}
\section{ADD, ADC and EFLAGS}\label{sec:sec1}
\subsection{The ADD instruction}\label{sec:subsec1}
\paragraph{}
Consider the following program:
\begin{Verbatim}[numbers=left, frame=single]
section .data

section .text
	global _start
	_start:
		MOV eax, 1
		MOV ebx, 2
		ADD eax, ebx
		INT 80h
\end{Verbatim}
The instruction \textbf{ADD eax, ebx} means the same as: \textbf{eax = eax + ebx}.

\subsubsection{Debugging}\label{sec:subsubsec1}
\paragraph{}
Below are the values of the registers \textbf{eax} and \textbf{ebx} in before and after performing addition.

\begin{figure}[h]
\centering
\includegraphics[width=0.5\textwidth]{Output35.png}
\caption{Value of \textbf{eax} after addition}
\label{fig:fig1}
\end{figure}

\begin{figure}[h]
\centering
\includegraphics[width=0.5\textwidth]{Output36.png}
\caption{Value of \textbf{ebx} before addition}
\label{fig:fig2}
\end{figure}

\begin{figure}[h]
\centering
\includegraphics[width=0.5\textwidth]{Output37.png}
\caption{Value of \textbf{eax} after addition}
\label{fig:fig3}
\end{figure}
\newpage
\subsection{EFLAGS}\label{sec:subsec2}
\paragraph{}
There is one more register called the \textbf{eflags} register. It is a $32$-bit register used to indicate the \textbf{status/information} about the computations and control CPU operations. Each bit represents a specific flag.

\begin{figure}[h]
\centering
\includegraphics[width=0.7\textwidth]{Output38.png}
\caption{The \textbf{eflags} register.}
\label{fig:fig4}
\end{figure}

Here \textbf{eflags} is giving information about the \textbf{ADD} operation and we see that two of its flags are set \textbf{PF} and \textbf{IF}.

\vspace{10pt}
\textbf{PF} is the \textbf{P}arity \textbf{F}lag. This flag is \textit{set} if the value of an operation is \textbf{odd}. If the value of an operation is an even number then PF is set to $0$. So, the \textbf{ADD} operation resulted into an odd value which is $3$ and that's why PF was set.

\vspace{10pt}
\textbf{IF} is the \textbf{I}nterrupt \textbf{F}lag. This flag is \textit{set} to $1$ when we allow interrupts to be done on our system.

The figure below shows the the \textbf{eflags} register:
\begin{figure}[h]
\centering
\includegraphics[width=0.8\textwidth]{EFLAGS.jpg}
\caption{The \textbf{eflags} register.[Image source: \href{https://www.geeksforgeeks.org/eflags-registers-of-80386-microprocessor/}{GeeksForGeeks}]}
\label{fig:fig5}
\end{figure}

\rule{\linewidth}{5pt}
\underline{\textbf{NOTE:}} In the assembly program above, include the line: \textbf{MOV eax, 1} before the interrupt statement otherwise the program won't exit and it will throw \textbf{segmentation fault}.

\rule{\linewidth}{5pt}
\newpage
\subsubsection{More flags of the EFLAGS register}\label{sec:subsubsec2}
\paragraph{}
Consider the following program:
\begin{Verbatim}[numbers=left, frame=single]
section .data

section .text
	MOV al, 0b11111111
	MOV bl, 0b0001
	ADD al, bl
	MOV eax, 1
	INT 80h
\end{Verbatim}
The notation \textbf{0b} indicates a binary number. Therefore \textbf{0b11111111} indicates a binary number. Basically, this program adds two numbers. One is $-1$(in \textbf{al}) and the other is $1$(in \textbf{bl}). We know that the result is $0$ but we want to see how this is operation performed and if any carry is generated then where that carry is stored.
\begin{figure}[h]
\centering
\includegraphics[width=0.6\textwidth]{Output39.png}
\caption{\textbf{al} stores $-1$}
\label{fig:fig6}
\end{figure}

\begin{figure}[h]
\centering
\includegraphics[width=0.5\textwidth]{Output40.png}
\caption{\textbf{bl} stores $1$}
\label{fig:fig7}
\end{figure}
\newpage
\begin{figure}[h]
\centering
\includegraphics[width=0.5\textwidth]{Output41.png}
\caption{\textbf{al} stores $0$}
\label{fig:fig8}
\end{figure}

\begin{figure}[h]
\centering
\includegraphics[width=0.5\textwidth]{Output42.png}
\caption{\textbf{eax} also stores $0$}
\label{fig:fig9}
\end{figure}

\begin{figure}[h]
\centering
\includegraphics[width=0.7\textwidth]{Output43.png}
\caption{\textbf{eflags} has some set bits}
\label{fig:fig10}
\end{figure}

The \textbf{ADD} operation looks like this:

%We can see from \hyperref[fig:

\begin{align*}
11111111 \\
00000001 \\
\_\_\_\_\_\_\_\_\_\_\_\_ \\
00000000
\end{align*}
This operation was shown for \textbf{al} and \textbf{bl}. So when we check the register \textbf{al}(\hyperref[fig:fig8]{Figure~\ref{fig:fig8}}), we see that it stores $0$. But interestingly enough, when we check the \textbf{eax} register(\hyperref[fig:fig9]{Figure~\ref{fig:fig9}}), we see that it also stores $0$.

The reason is that, the \textbf{ADD} operation was performed for only $1$ byte registers(\textbf{al} and \textbf{bl} in this case). So, when the extra $1$(carry) resulted from the addition, it was put into the \textbf{CF} flag of the \textbf{eflags} register and that is the reason why we see $0$ as the value of \textbf{eax} as well.

\vspace{10pt}
\textbf{CF} is the \textbf{C}arry \textbf{F}lag. When there is a carry from the previous operation, then it's set to $1$.

\vspace{10pt}
\textbf{ZF} is the \textbf{Z}ero \textbf{F}lag. It is set if the register that results from the operation is $0$. In our case, \textbf{al} got set to $0$.
\newpage
\subsection{The ADC instruction}\label{sec:subsec3}
\paragraph{}
Consider this assembly program:
\begin{Verbatim}[numbers=left, frame=single]
section .data

section .text
	global _start
	_start:
		MOV al, 0b11111111
		MOV bl, 0b0001
		ADD al, bl  ; al = al + bl
		ADC ah, 0
		MOV eax, 1
		INT 80h

\end{Verbatim}
ADC instruction is called add with carry. It works similar to the \textbf{ADD} operation. It takes a destination and adds a source to it. It then adds the CF to the destination.

In this program we are using the \textbf{ah}(higher $8$ bits) to store the carry. The instruction \textbf{ADC ah, 0} will basically add $0$ to \textbf{ah} and then will also add the CF(the carry bit in CF) to it and store the result in \textbf{ah}.

\begin{figure}[h]
\centering
\includegraphics[width=0.6\textwidth]{Output44.png}
\caption{Value of \textbf{ah} and \textbf{eax}}
\label{fig:fig11}
\end{figure}

We can see that \textbf{ah} stores $1$(which is the CF) and \textbf{eax} stores $256$.

Now the reason \textbf{eax} stores $256$ is that it is a $32$-bit register and \textbf{ah} is its higher $8$-bits. So we know that the lower $8$ bits(\textbf{al}) were all zeros(from addition) and the higher $8$-bits were all $0$ as well(from the previous addition). But now, we used the \textbf{ADC} instruction to put the carry into the higher $8$ bits of \textbf{eax} which make it look like this:
$$
00000000\ 00000000\ 00000001\ 00000000
$$

which is $256$ in decimal.

\end{document}