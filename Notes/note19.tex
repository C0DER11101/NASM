\documentclass{article}
\usepackage{hyperref}
\usepackage{graphicx}
\usepackage{listings}
\usepackage{amsmath}
\usepackage{fancyvrb}

\hypersetup{
	colorlinks=true,
	linkcolor=red,
	urlcolor=blue
}

\makeatletter

\renewcommand\paragraph{\@startsection{paragraph}{4}{\z@}{-3.25ex \@plus -1ex \@minus -.2ex}{1.5ex \@plus .2ex}{\normalfont\normalsize\bfseries}}

\makeatother

\begin{document}
	\section{Opening and reading files}\label{sec:sec1}
	\paragraph{}
	Simple assembly program:
	\begin{lstlisting}[numbers=left, frame=single, breaklines=true]
section .data
	pathname DD "/home/priyanuj/temp/test.txt"
section .text
	global main
	
	main:
		MOV eax, 5
		MOV ebx, pathname
		MOV ecx, 0
		INT 80h
		
		MOV eax, 1
		INT 80h
	\end{lstlisting}
We are opening a text file for reading in the program above. $5$ is the code that will be stored in \textbf{eax}, \textbf{ebx} will store the pathname and \textbf{ecx} will store the mode in which the file will be opened. Here $0$ indicates that file is opened in read-only mode.

\subsection{Executing using GDB}\label{sec:ssec1}
\paragraph{}
We now, execute the program using GDB. First, we assemble and compile the program using \textbf{nasm} and \textbf{gcc}.
\begin{figure}[h]
	\centering
	\includegraphics[width=1.1\textwidth]{Output115.png}
	\caption{Code and address of pathname moved into \textbf{eax} and \textbf{ebx} respectively}
	\label{fig:fig1}
\end{figure}
\newpage
We can see the values in \textbf{eax} and \textbf{ebx} as follows:
\begin{figure}[h]
	\centering
	\includegraphics[width=0.9\textwidth]{Output116.png}
	\caption{Values in \textbf{eax} and \textbf{ebx}}
	\label{fig:fig2}
\end{figure}

\textbf{ebx} stores the address of the \textbf{pathname}. We can view how the pathname is stored as:
\begin{Verbatim}[frame=single]
(gdb) x/10x 0x804c018
\end{Verbatim}

This will give us the following output:
\begin{figure}[h]
	\centering
	\includegraphics[width=1.2\textwidth]{Output117.png}
	\caption{String stored in addresses}
	\label{fig:fig3}
\end{figure}

We can also view the actual string as:
\begin{Verbatim}[frame=single]
(gdb) x/10s 0x804c018
\end{Verbatim}
We see the following output:
\begin{figure}[h]
	\centering
	\includegraphics[width=0.9\textwidth]{Output118.png}
	\caption{Pathname stored in memory}
	\label{fig:fig4}
\end{figure}
\newpage
Now, after executing the interrupt \textbf{INT 80h}, we can see that the value of \textbf{eax} has changed:
\begin{figure}[h]
	\centering
	\includegraphics[width=0.7\textwidth]{Output119.png}
	\caption{\textbf{eax} changed}
	\label{fig:fig5}
\end{figure}

Now, after the performing the interrupt, a file descriptor(an integer) is returned in \textbf{eax}. Here the file descriptor is $3$.

\newpage
\section{Reading contents from the file}\label{sec:sec2}
\paragraph{}
We can use this file descriptor to read data from a file.
\begin{lstlisting}[frame=single, numbers=left, breaklines=true]
section .data
	pathname DD "/home/priyanuj/temp/test.txt"

section .bss
	buffer RESB 1024

section .text
	MOV eax, 5
	MOV ebx, pathname
	MOV ecx, 0
	INT 80h
	
	MOV ebx, eax
	MOV eax, 3
	MOV ecx, buffer
	MOV edx, 1024
	INT 80h
	
	MOV eax, 1
	INT 80h
\end{lstlisting}
In order to save the read data, we need a buffer, so we declared an uninitialized buffer of $1024$ bytes. Now, the file descriptor is stored in \textbf{eax}, so we need to move it to \textbf{ebx} first so that we can store the code $3$ into \textbf{eax} which is the code for \textbf{sys\_read} so that we can read from the file.
\newpage
The image below shows the execution of the program:
\begin{figure}[h]
	\centering
	\includegraphics[width=1.1\textwidth]{Output120.png}
	\caption{Executing}
	\label{fig:fig6}
\end{figure}

We can see an address \textbf{0x804c038} which is being moved into \textbf{ecx}. That is actually the address of the buffer \textbf{buffer}.

\begin{figure}[h]
	\centering
	\includegraphics[width=0.8\textwidth]{Output121.png}
	\caption{Value in \textbf{eax} and the contents of \textbf{0x804c038}}.
	\label{fig:fig7}
\end{figure}

So, the we can see that \textbf{eax} stores the number $44$. The number actually represents the number of characters read from the file. So, it read $44$ characters from the file because there were $44$ characters in the file.

The string is stored in the memory slot starting from the address \textbf{0x804c038} which we can see from the image above.
\end{document}