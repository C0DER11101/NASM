\documentclass{article}
\usepackage{hyperref}
\usepackage{fancyvrb}
\usepackage{graphicx}

\hypersetup{
	colorlinks=true,
	linkcolor=red,
	urlcolor=blue
}

\makeatletter

\renewcommand\paragraph{\@startsection{paragraph}{4}{\z@}{-3.25ex \@plus -1ex \@minus -.2ex}{1.5ex \@plus .2ex}{\normalfont\normalsize\bfseries}}

\makeatother

\begin{document}

\section{Characters, strings and lists}
\subsection{Characters}\label{sec:chars}
\paragraph{}
Consider the following program:
\begin{Verbatim}[numbers=left, frame=single]
section .data
	char DB 'A'
	
section .text
	global _start
	_start:
		MOV bl, [char]
		MOV eax, 1
		INT 80h
\end{Verbatim}

The value of \textbf{\$?} is as follows:
\begin{figure}[h]
\centering
\includegraphics[width=0.9\textwidth]{Output26.png}
\caption{Value of \textbf{\$?}}
\label{fig:statusCode}
\end{figure}

Since \textbf{bl} is a sub-register of the register \textbf{ebx} and the register \textbf{ebx} is used to store status code which is stored in \textbf{\$?} that's why we can see $65$ in \textbf{\$?}.

\vspace{10pt}
We know that $65$ is the \href{https://www.cs.cmu.edu/~pattis/15-1XX/common/handouts/ascii.html}{ASCII value} of the character \textbf{A}. When we work with anyting in assembly, it's stored as numeric(mostly as binary) data. So, when storing the character \textbf{A}(or any other character) it is encoded in a way so that it has a numeric value(there are many encodings like ASCII, UTF-8, etc) that represents \textbf{A}(or any other character). So, here $65$ represents the character \textbf{A}.
\newpage
\subsection{Lists}\label{sec:lists}
\paragraph{}
Consider this program:

\begin{Verbatim}[numbers=left, frame=single]
section .data
	list DB 1, 2, 3, 4

section .text
	global _start
	_start:
		MOV bl, [list]
		MOV eax, 1
		INT 80h
\end{Verbatim}
Here, \textbf{list} is a list of bytes.

Let's see how the bytes are stored in the memory.

\begin{figure}[h]
\centering
\includegraphics[width=0.5\textwidth]{Output27.png}
\caption{Bytes in \textbf{0x804a000}}
\label{fig:bytes}
\end{figure}
We can see the bytes \textbf{0x01}, \textbf{0x02}, \textbf{0x03} and \textbf{0x04}. So, basically it's a list of bytes. Declaring a list of bytes is similar to declaring separate bytes(as seen from \href{https://github.com/C0DER11101/NASM/blob/nAsM/Notes/note3.pdf}{here}). They are stored in the same manner.

\vspace{10pt}
Now, we will run the following commands:
\begin{Verbatim}[frame=single]
(gdb) x/x 0x804a000
(gdb) x
\end{Verbatim}

Here is the output:
\begin{figure}[h]
\centering
\includegraphics[width=0.5\textwidth]{Output28.png}
\caption{\textbf{x/x} \textbf{0x804a000}}
\label{fig:x/x}
\end{figure}

From the output it's clear that \textbf{x/x 0x804a000} showed the contents of the memory block starting from \textbf{0x804a000} till \textbf{0x804a003}(Visit this \href{https://visualgdb.com/gdbreference/commands/x}{link}).
\newpage
When we use only the \textbf{x}(no address expression) command, it basically displays the memory contents from the address following the contents of the last memory address. In this case, we can see that \textbf{x/x 0x804a000} printed the contents of memory addresses \textbf{0x804a000}, \textbf{0x804a001}, \textbf{0x804a002} and \textbf{0x804a003}. Next when we run the only the \textbf{x} command then it displays the contents inside memory addresses starting from \textbf{0x804a004} all the way till \textbf{0x804a007}.

The following command:
\begin{Verbatim}[frame=single]
(gdb) x/2x 0x804a000
\end{Verbatim}
will display the contents of $2$ units of memory ($4$ bytes/unit). This means the contents of the first memory unit(from \textbf{0x804a000} to \textbf{0x804a003}) will be displayed and then the contents of second memory unit(from \textbf{0x804a004} to \textbf{0x804a007}) will be displayed.
\newpage
\subsection{Strings}\label{sec:strings}
\paragraph{}
The following x86 assembly program uses strings:

\begin{Verbatim}[numbers=left, frame=single]
section .data
	string1 DB "ABA", 0
	string2 DB "CDE", 0
	
section .text
	global _start
	_start:
		MOV bl, [string1]
		MOV eax, 1
		INT 80h
\end{Verbatim}
The $0$ at the end of each string is actually the \textbf{\textit{null terminator}}.

We run the following command:
\begin{Verbatim}[frame=single]
(gdb) x/x 0x804a000
\end{Verbatim}

The following output is generated:
\begin{figure}[h]
\centering
\includegraphics[width=0.5\textwidth]{Output29.png}
\caption{Storing strings}
\label{fig:storeStrings}
\end{figure}

We can see the hex value of \textbf{A}(\textbf{0x41}), of \textbf{B}(\textbf{0x42}) and of \textbf{A}(\textbf{0x41}).

Now, again, we will use the \textbf{x/2x} command on \textbf{0x804a000}. We get the following output:

\begin{figure}[h]
\centering
\includegraphics[width=0.6\textwidth]{Output30.png}
\caption{Output of \textbf{x/2x 0x804a000}}
\label{fig:x/2x}
\end{figure}

We have two strings and each has a null terminator.

From memory blocks starting from \textbf{0x804a000} to \textbf{0x804003} we have stored the string \textbf{"ABA$\backslash$0"}. The \textbf{0x00} is actually the null terminator(not an empty or uninitialized byte). The next hexadecimal value \textbf{0x00454443} is actually the string \textbf{"CDE$\backslash$0"}(\textbf{0x00} is for the null terminator("$\backslash$0")).

\newpage
These two strings are stored in two different units. $1$ unit starts from \textbf{0x804a000} to \textbf{0x804a003}($4$ bytes) and the other unit starts from \textbf{0x804a004} to \textbf{0x804a007}($4$ bytes).

\end{document}