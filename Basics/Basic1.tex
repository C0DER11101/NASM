\documentclass{article}
\usepackage{amsmath}

\makeatletter

\renewcommand\paragraph{\@startsection{paragraph}{4}{\z@}{-3.25ex \@plus -1ex \@minus -.2ex}{1.5ex \@plus .2ex}{\normalfont\normalsize\bfseries}}

\makeatother

\begin{document}

\section{Binary operations}
\paragraph{\underline{Addition}}

\begin{align*}
0 + 0 &= 0 \\
0 + 1 &= 1 \\
1 + 0 &= 1 \\
1 + 1 &= 10 \\
\end{align*}

$1\ +\ 1\ = 2$ but since we are working with binary here, we write $(2)_{10}$ as $(10)_2$.

\paragraph{\underline{Signed binary numbers}}

To represent a negative number in binary, we apply $2$'s complement.

Consider the decimal number $2$.

The $32$-bit representation of $(2)_{10}$ is:
$$
(00000000\ 00000000\ 00000000\ 00000010)_2
$$

To find the $2$'s complement, we \textbf{first invert all the bits of the binary representation of} $2$. And this what we get:

$$
(11111111\ 11111111\ 11111111\ 11111101)_2
$$

This is $1$'s complement of $2$.

Now, for $2$'s complement, we \textbf{add a $1$ to this $1$'s complement}.

\begin{align*}
11111111 11111111 11111111 11111101 \\
                                 +1 \\
\hline
11111111 11111111 11111111 11111110
\end{align*}


So, $(11111111 11111111 11111111 11111110)_2$ is the binary representation of $-2$.

Now, if we add $-2$ and $2$, we end up with $0$.

This condition is satisfied by their binary representations as well.

\begin{align*}
 00000000 00000000 00000000 00000010 \\
+11111111 11111111 11111111 11111110 \\
\hline
 00000000 00000000 00000000 00000000
\end{align*}

Now, notice there that we still have a $1$ as a carry. In assembly languages, this $1$ goes into a special register known as \textbf{carry register}.

\end{document}