\documentclass{article}
\usepackage{hyperref}
\usepackage{graphicx}
\usepackage{amsmath}
\usepackage{fancyvrb}

\hypersetup{
	colorlinks=true,
	linkcolor=red,
	urlcolor=blue
}

\makeatletter

\renewcommand\paragraph{\@startsection{paragraph}{4}{\z@}{-3.25ex \@plus -1ex \@minus -.2ex}{1.5ex \@plus .2ex}{\normalfont\normalsize\bfseries}}

\makeatother

\begin{document}
	\section{Floating point numbers}\label{sec:sec1}
	\paragraph{}
	Consider the following program:
	\begin{Verbatim}[numbers=left, frame=single]
		section .data
			x DD 3.14
			y DD 2.1
			
		section .text
			MOVSS xmm0, [x]
			MOVSS xmm1, [y]
			
			ADDSS xmm0, xmm1
			
			MOV eax, 1
			INT 80h
	\end{Verbatim}
See about \textbf{xmm} registers \href{https://www.oreilly.com/library/view/mastering-assembly-programming/9781787287488/50685a1c-0812-407c-8d7d-d7a9202722b3.xhtml}{here}. 

\vspace{10pt}
\href{https://stackoverflow.com/questions/10025360/how-many-xmm-registers-are-available-on-an-x86-processor-supporting-sse}{How many \textbf{xmm} registers?}.

\vspace{10pt}
\textbf{MOVSS} instruction is used to move the floating point values to the \textbf{xmm} registers. \textbf{SS} means \textbf{S}caler \textbf{S}ingle precision.

\vspace{10pt}
We will execute this program with GDB.
\begin{figure}[h]
	\centering
	\includegraphics[width=1.35\textwidth]{Output90.png}
	\caption{Values in \textbf{xmm}}
	\label{fig:fig1}
\end{figure}

We can see a bunch of arrays(I guess) which have a bunch of values in them.

To get the value stored in \textbf{xmm0} we particularly use the \textbf{v4\_float} array.
\begin{figure}[h]
	\centering
	\includegraphics[width=0.5\textwidth]{Output91.png}
	\caption{\textbf{v4\_float} array's $0^{th}$ element(\textbf{xmm0})}
	\label{fig:fig2}
\end{figure}

Notice that we assigned the value $3.14$ to \textbf{xmm0} but in this image it's showing $3.1400001$.
\newpage
Similar behaviour can be seen in \textbf{xmm1} as well.
\begin{figure}[h]
	\centering
	\includegraphics[width=0.5\textwidth]{Output92.png}
	\caption{\textbf{v4\_float} array's $0^{th}$ element(\textbf{xmm1})}
	\label{fig:fig3}
\end{figure}

The reason why these values are not stored exactly as they were defined is that we use a special notation called \href{https://staffwww.fullcoll.edu/aclifton/cs241/lecture-floating-point-simd.html}{IEEE} floating-point notations.

From the image below:
\begin{figure}[h]
	\centering
	\includegraphics[width=0.5\textwidth]{Output93.png}
	\caption{Sum of \textbf{xmm0} and \textbf{xmm1}}
	\label{fig:fig4}
\end{figure}

We can see that the sum is also not exact.
\end{document}