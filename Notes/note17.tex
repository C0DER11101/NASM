\documentclass{article}
\usepackage{hyperref}
\usepackage{graphicx}
\usepackage{listings}
\usepackage{amsmath}
\usepackage{fancyvrb}

\hypersetup{
	colorlinks=true,
	linkcolor=red,
	urlcolor=blue
}

\makeatletter

\renewcommand\paragraph{\@startsection{paragraph}{4}{\z@}{-3.25ex \@plus -1ex \@minus -.2ex}{1.5ex \@plus .2ex}{\normalfont\normalsize\bfseries}}

\makeatother

\begin{document}
	\section{Basics of functions/procedures in x86}\label{sec:sec1}
	\paragraph{}
	Assembly program:
	\begin{lstlisting}[frame=single, breaklines=true]
		section .data
		
		section .text
			global main
			
			addTwo:
				ADD eax, ebx
				RET
			
			main:
				MOV eax, 4
				MOV ebx, 1
				
				CALL addTwo
				
				MOV ebx, eax
				
				MOV eax, 1
				INT 80h
	\end{lstlisting}
\textbf{addTwo} is a procedure/function that basically \textbf{ADD}s two registers and \textbf{RET}urns the sum.

\vspace{10pt}
Debugging the program using GDB and observing the stack. \\
\textbf{ESP} $\rightarrow$ \textbf{E}xtended \textbf{S}tack \textbf{P}ointer. This register contains the address of the top of the stack.

The control flow is very simple in this program. We have two registers \textbf{ESP} and \textbf{EIP}(the instruction pointer). The \textbf{EIP} always stores the address of(or points to) the next instruction to be executed. It's like the program counter(\textbf{pc} register) in MIPS.

\vspace{10pt}
Now, when \textbf{addTwo} is \textbf{CALL}ed i.e when the instruction \textbf{CALL addTwo} is executed, then \textbf{EIP} was pointing to the next instruction that is \textbf{MOV ebx, eax}. Now, \textbf{ESP} was decremented(\textbf{stack grows downwards}) and the address that \textbf{EIP} was holding gets pushed onto the stack. So, the address that \textbf{EIP} was holding is the new top of the stack and \textbf{ESP} points to this new top. Now, since the address held by \textbf{EIP} gets stored in the stack, the address of the instruction \textbf{add eax, ebx} gets stored in \textbf{EIP}. Now, the body of \textbf{addTwo} starts getting executed. When \textbf{addTwo} \textbf{RET}urns then \textbf{ESP} is incremented, that is the the address of \textbf{MOV ebx, eax} is popped from the stack and restored into \textbf{EIP} and so the we see the lines after the \textbf{CALL} instruction execute normally.
\newpage
Now, we just executed the first line inside our \textbf{main} procedure(rather label).
\begin{figure}[h]
	\centering
	\includegraphics[width=0.9\textwidth]{Output98.png}
	\caption{Executing the first line}
	\label{fig:fig1}
\end{figure}

If we inspect the \textbf{EIP} we will see that it stores the address of the next instruction to be executed(the highlighted instruction in the image above).
\begin{figure}[h]
	\centering
	\includegraphics[width=0.7\textwidth]{Output97.png}
	\caption{Address pointed to by \textbf{EIP}}
	\label{fig:fig2}
\end{figure}

Also, \textbf{ESP} is currently storing some address(which is not known to us), that is basically it is pointing to whatever is in the top of the stack at the moment.
\newpage
Now, we executed the next instruction:
\begin{figure}[h]
	\centering
	\includegraphics[width=1.2\textwidth]{Output99.png}
	\caption{Executing}
	\label{fig:fig3}
\end{figure}

We can see that the value of \textbf{EIP} is the address of the next instruction(the highlighted one) and \textbf{ESP} is still unchanged.
\begin{figure}[h]
	\centering
	\includegraphics[width=0.9\textwidth]{Output100.png}
	\caption{\textbf{EIP} and \textbf{ESP}}
	\label{fig:fig4}
\end{figure}

\newpage
Let's execute the next line:
\begin{figure}[h]
	\centering
	\includegraphics[width=1.2\textwidth]{Output101.png}
	\caption{\textbf{addTwo} about to be executed}
	\label{fig:fig5}
\end{figure}

We can see that the first instruction of \textbf{addTwo} is about to get executed. Let's see what has changed in \textbf{EIP} and \textbf{ESP}:
\begin{figure}[h]
	\centering
	\includegraphics[width=0.9\textwidth]{Output102.png}
	\caption{Changes in \textbf{EIP} and \textbf{ESP}}
	\label{fig:fig6}
\end{figure}

We can see that \textbf{EIP} stores the address of the first instruction of \textbf{addTwo} procedure and \textbf{ESP} stores \textbf{0xffffcf48}. So the value in \textbf{ESP} decreased by $4$.
\newpage
Now, if we try to view what is in this address pointed to by \textbf{ESP} then we will get:
\begin{figure}[h]
	\includegraphics[width=0.7\textwidth]{Output103.png}
	\centering
	\label{fig:fig7}
\end{figure}

We can see that the address that we get after using the \textbf{x/x} command is the address of the instruction \textbf{mov ebx, eax} which is the instruction right after \textbf{CALL addTwo} instruction as seen from \hyperref[fig:fig5]{Figure~\ref{fig:fig5}}.

Now, we are in the \textbf{RET}urn instruction of \textbf{addTwo}:
\begin{figure}[h]
	\centering
	\includegraphics[width=1.2\textwidth]{Output104.png}
	\caption{\textbf{RET}}
	\label{fig:fig8}
\end{figure}
\newpage
We execute this line. After that, we can see that the instruction \textbf{mov ebx, eax} is the next instruction to be executed.
\begin{figure}[h]
	\centering
	\includegraphics[width=1.2\textwidth]{Output106.png}
	\caption{After \textbf{CALL}}
	\label{fig:fig9}
\end{figure}

We can also see that \textbf{ESP} is now restored and \textbf{EIP} stores the address of the instruction \textbf{mov ebx, eax}.
\begin{figure}[h]
	\centering
	\includegraphics[width=0.8\textwidth]{Output105.png}
	\caption{\textbf{ESP} and \textbf{EIP}}
	\label{fig:fig10}
\end{figure}

This \href{https://youtu.be/RU5vUIl1vRs?si=4SnJANio6Tap4lX4}{video} helped me a lot in understanding the stack.
\newpage
\section{Sources}
\begin{itemize}
	\item \url{https://en.wikibooks.org/wiki/X86_Disassembly/The_Stack}
	\item \url{https://wiki.osdev.org/Stack}
	\item \url{https://www.youtube.com/watch?v=vcfQVwtoyHY}
	\item \url{https://youtu.be/F58WAnf2gr0?si=Bj4B00pnJ7N0rG5v}
	\item \href{https://stackoverflow.com/questions/40324514/what-is-the-difference-between-esp-and-eip-registers}{https://stackoverflow.com/questions/40324514/what-is-the-difference-between-esp-and-eip-registers}
	\item \url{https://stackoverflow.com/questions/3699283/what-is-stack-frame-in-assembly}
\end{itemize}
\end{document}