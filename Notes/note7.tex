\documentclass{article}
\usepackage{amsmath}
\usepackage{hyperref}
\usepackage{graphicx}
\usepackage{fancyvrb}

\hypersetup{
	colorlinks=true,
	linkcolor=red,
	urlcolor=blue
}

\makeatletter

\renewcommand\paragraph{\@startsection{paragraph}{4}{\z@}{-3.25ex \@plus -1ex \@minus -.2ex}{1.5ex \@plus .2ex}{\normalfont\normalsize\bfseries}}

\makeatother

\begin{document}
\section{Subtraction and sign flags}\label{sec:sec1}
\subsection{Subtraction}\label{sec:subsec1}
\paragraph{}
Assembly program:
\begin{Verbatim}[numbers=left, frame=single]
section .data

section .text
	MOV eax, 5
	MOV ebx, 3
	SUB eax, ebx
	MOV eax, 1
	INT 80h
\end{Verbatim}
The instruction \textbf{SUB eax, ebx} means $eax\ =\ eax\ -\ ebx$.
This is normal subtraction where we are subtracting a smaller number from a larger number.

We can see the values of the register \textbf{eax} before and after subtraction.

\begin{figure}[h]
\centering
\includegraphics[width=0.5\textwidth]{Output45.png}
\caption{Values in \textbf{eax}}
\label{fig:fig1}
\end{figure}

This was simple subtraction.

Now, we will execute the following assembly program using GDB:
\begin{Verbatim}[numbers=left, frame=single]
section .data

section .text
	MOV eax, 3
	MOV ebx, 5
	SUB eax, ebx
	MOV eax, 1
	INT 80h
\end{Verbatim}

Here also, we are performing the subtraction operation, but this time we are subtracting the larger number from the smaller number(which will result in a negative number).
\newpage
We can see that \textbf{eax} stores $-2$.
\begin{figure}[h]
\centering
\includegraphics[width=0.5\textwidth]{Output46.png}
\caption{Negative value in \textbf{eax}}
\label{fig:fig2}
\end{figure}

We can also view which flags were set during this operation.
\begin{figure}[h]
\centering
\includegraphics[width=0.6\textwidth]{Output47.png}
\caption{The flags that were set}
\label{fig:fig3}
\end{figure}

The \textbf{CF} serves two functions in x86:
\begin{itemize}
\item it represents a carry
\item it also represents a \textbf{borrow}
\end{itemize}

Following are some rules for binary subtraction:
\begin{align*}
0 - 0 &= 0 \\
1 - 0 &= 1 \\
0 - 1 &= 1(with\ borrow\ 1) \\
1 - 1 &= 0 \\
\end{align*}

\vspace{10pt}
\textbf{SF} means the \textbf{S}ign \textbf{F}lag. When this flag is set to $1$, it indicates that the operation produced a negative output(in our case it's $-2$).

Let's say that we added the following lines to the assembly program above:
\begin{Verbatim}[numbers=left, frame=single]
MOV ebx, 2
ADD eax, ebx
\end{Verbatim}
\newpage
Now, what we basically did was this: $-2$ was stored in the register \textbf{eax}. Now we changed the value of \textbf{ebx} to $2$ and then performed addition operation on \textbf{eax} and \textbf{ebx}. Since we are adding $-2$(in \textbf{eax}) and $2$(in \textbf{ebx}) we get a $0$ which is stored in register \textbf{eax}.

\begin{figure}[h]
\centering
\includegraphics[width=0.5\textwidth]{Output48.png}
\caption{Values of \textbf{eax} before and after addition}
\label{fig:fig4}
\end{figure}

\end{document}